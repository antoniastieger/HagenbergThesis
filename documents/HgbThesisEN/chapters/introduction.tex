\chapter{Introduction}
\label{cha:Introduction}

\section{Motivation}
In today's world, climate change is a pressing global issue, driving efforts to reduce carbon footprints.
One effective strategy to mitigate environmental impacts is increasing public transportation use over private vehicles.
Despite the clear environmental advantages, such as reduced fuel consumption and lower emissions, many commuters still favor private cars due to their convenience and flexibility.
According to the United Nations \cite{un2023}, switching to public transport could reduce an individual's carbon footprint by up to 2.2 tons annually.

However, as Zheng and Krol \cite{mit2023} note, widespread adoption will only happen if public transit becomes the “most convenient option for getting around.”
While service coverage and infrequent schedules are well-known barriers, another common issue is that passengers often miss their stop due to distractions.
Whether listening to music, reading or using their smartphones, commuters can easily lose track of their location.
To address this issue, this thesis explores the development of a location-based alert system that notifies users as they approach their destination.
By helping passengers stay aware of their stop, such a system could make public transportation more accessible and user-friendly, ultimately increasing rates of usage.

\section{Challenges}
One significant challenge to this goal is the strong competition from private vehicles, particularly in suburban and rural regions, where car ownership is widespread, and public transport struggles to provide a viable alternative. 
While urban areas often benefit from more comprehensive public transport networks, less populated regions face specific difficulties with limited routes, infrequent schedules and longer distances between stops.
In these areas, missing a stop can lead to long waits for the next transport or even stranding passengers.

Beyond challenges related to public transport, several technological challenges arise in developing an effective alert system. 
Apple's app development guidelines impose strict restrictions on background activity, limiting how and when alerts can be triggered. 
Geofencing requires continuous location tracking, which impacts battery life and may raise privacy concerns, as noted by Shevchenko and Reips \cite{shevchenko2023geofencing}.
Additionally, defining an appropriate geofence radius is important, as a poorly chosen size can cause inaccuracy or latency.
Some devices may lack real-time location capabilities, meaning geofencing might not work reliably or at all for those users.

\section{Goals}
This thesis aimes to develop a user-focused app-based alert system in iOS that integrates schedule data and geofencing. 
Passengers are provided with a public transport app that will notify them along their journey when their destination approaches.
This approach, particularly valuable in rural areas, ensures that regular commuters as well as tourists, children, and the elderly can navigate public transportation networks with ease.

The app offers users the flexibility to choose from three distinct alert modes based on their preferences. 
The first mode sends alerts according to the scheduled arrival times of transports. 
The second uses three geofences along the route to progressively notify passengers as they near their desired stop. 
The third relies on nested geofences around the final destination to trigger alerts. 
Additionally, the user will be alerted with three escalating levels of urgency, ensuring passengers are notified effectively.
Furthermore, the app aims develop a test feature for wizard-based design that uses voice control for setting up reminders.

\section{Structure}
Before diving into the implementation, Chapter 2 introduces the core technologies relevant to this project, focusing on positioning methods and the systems that implement them. 
The chapter also discusses alerts on smartphones and their restrictions within iOS. 
Chapter 3 analyzes existing public transportation apps and how they inform users of approaching destinations.
Chapter 4 presents the app concept, detailing the prototype design, proposed reminder behavior and data persistence strategy.
Chapter 4 discusses the implementation in iOS, covering system architecture, geofence management, reminder triggering and voice-controlled alert setup. 
This thesis closes with Chapter 5, which describes what has been achieved and outlines potential next steps.